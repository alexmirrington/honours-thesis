\usepackage{bm}
\usepackage{geometry}
  \geometry{a4paper,inner=3cm, outer=3cm, top=3cm, bottom=3cm}
\pdfpagewidth=\paperwidth 
\pdfpageheight=\paperheight

\usepackage[parfill]{parskip} 
% Activate to begin paragraphs with an empty (return) line, comment out the indent below if you chose the return line option.

% \setlength{\parindent}{1em}  % Sets the length of the paragraph indent. Current setup has a an indent. Disable this if you activate the return line above.

% Double or one and a half spacing.
%\usepackage{setspace}
% \singlespacing

  
\usepackage{graphicx}
\DeclareGraphicsRule{.tif}{png}{.jpg}{`convert #1 `dirname #1`/`basename #1 .tif`.png}
\graphicspath{{figures/}}
% Graphics. Remove me and you won't have any figures, and that would be very boring.

%\usepackage[usenames,dvipsnames,svgnames,table]{xcolor}
% Adds the ability to make coloured text and lines throughout the document. See documentation for xcolor.

%-------------------- Tables, figures and captions
\usepackage[font={small},labelfont={bf},margin=4ex]{caption}
% Makes bold labeled and smaller font captions. Must be loaded before the longtable package to avoid conflicts! 

\usepackage{longtable}
% Long tables (more than one page). Different headers and footers for beginning and end pages, etc.

\usepackage{afterpage}
% Make a longtable start on the next clear page, but fills the previous one with text first (no random gaps in the text-from long tables anymore! Man, the day I discovered this...)

\usepackage{booktabs}
% Nice looking tables and lines in tables

\usepackage{multirow}
% Entries in tables over multiple rows

\usepackage{lscape}
% Pages in landscape

\usepackage{pdflscape}
% Landscape pages also rotated in the pdf

\usepackage{wrapfig}
% Allows figures that don't take up the entire width of the page, wrapping the text around the figure

%\usepackage[position=top,singlelinecheck=false,captionskip=4pt]{subfig} 
% Multiple figures in an individual figure. Fig. 1 a, b, c, etc. each with, or without, it's own individual caption, and with a global caption for all sub figures.

%-------------------- Special symbols and fonts
\usepackage{amssymb}
\usepackage{amsfonts}
\usepackage{amsmath}
% Maths symbols

%-------------------- Document sections, headers, footers, and bibliography
\usepackage{fancyhdr}

%-------------------- Bibliography

\usepackage[backend=biber,style=ieee,doi=false]{biblatex}

% The hyperref package allows you to have clickable links in your pdf. It also allows you to have the mailto link associated with your name. It can be  a bit finicky, so load it last.
\usepackage{hyperref}
\hypersetup{
	colorlinks   = true,
	allcolors    = black
}

% Command renewals, New commands etc.
\renewcommand{\thefootnote}{\arabic{footnote}}

% More packages
\usepackage{tabularx}
\usepackage{xltabular}
\newcolumntype{C}{>{\centering\arraybackslash}X}
\newcolumntype{L}{>{\raggedright\arraybackslash}X}
\newcolumntype{R}{>{\raggedleft\arraybackslash}X}

\usepackage{float}
\usepackage{caption}
\usepackage{subcaption}

\usepackage{forest}