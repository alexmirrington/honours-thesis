% Grouping
\renewcommand\nomgroup[1]{%
  \item[\bfseries
  \ifstrequal{#1}{D}{Data}{%
  \ifstrequal{#1}{M}{Metrics}{%
  \ifstrequal{#1}{G}{Graphs}{%
  \ifstrequal{#1}{X}{General}{
  \ifstrequal{#1}{A}{Model Architecture}{}}}}}%
]}

% General
\nomenclature[X]{\(I_n\)}{An identity matrix of size \(n \times n\).}

\nomenclature[X]{\([a \cat b]\)}{The concatenation of two vectors \(a\) and \(b\).}

% Data
\nomenclature[D, 00]{\(S = \{((q_i, r_i), a_i)\}_{i=1}^n\)}{A visual question answering (VQA) dataset of \(n\) samples, each comprising a question \(q\), a visual representation \(r\) in which the question is grounded, and an answer \(a\).}

% \nomenclature[D, 00]{\(B = \{((q_j, r_j), a_j)\}_{j=i}^{i+k}\)}{A subset or mini-batch of a (VQA) dataset \(S\). \(B\) has size }

\nomenclature[D, 01]{\(q\)}{A visually-grounded question.}

\nomenclature[D, 02]{\(r\)}{A visual representation in which a question is grounded. Since different VQA datasets contain one or more visual representations for each dataset sample (\textit{e.g.} an image, a scene graph, a set of spatial or object-based features, or a combination of these), I specify the kind of data \(r\) represents for a given context.}

\nomenclature[D, 03]{\(a\)}{A ground-truth answer to a visually-grounded question}

\nomenclature[D, 04]{\(\hat{a}\)}{A predicted answer to a visually-grounded question.}

\nomenclature[D, 05]{\(Q = \{q \mid ((q, s), y) \in S\}\)}{The collection of all questions in a dataset \(S\).}

\nomenclature[D, 06]{\(Q_{[i:i+k]} = \{q_j \mid ((q_j, s_j), y_j) \in S\}_{j=i}^{\min(i + k, n)}\)}{A batch of questions of size \(\min(k, n - i)\) from a dataset \(S\) of size \(n\), starting at index \(i\).}

\nomenclature[D, 07]{\(R = \{r \mid ((q, r), y) \in S\}\)}{The collection of all visual representations in a dataset \(S\).}

\nomenclature[D, 08]{\(R_{[i:i+k]} = \{r_j \mid ((q_j, r_j), y_j) \in S\}_{j=i}^{\min(i + k, n)}\)}{A batch of visual representations of size \(\min(k, n - i)\) from a dataset \(S\) of size \(n\), starting at index \(i\). For readability, I refer to the size of the batch dimension of tensors as \(k\) even though smaller batches of size \(\min(k, n - i)\) may occur.}

% Model

\nomenclature[A, 00]{\(H_{\cdot}\)}{An embedding matrix for some entity \(\cdot\), \textit{e.g.} a visual representation \(r\), a question 
\(q\) or batches thereof.}

\nomenclature[A, 01]{\(K_{\cdot}\)}{A knowledge-base for some entity \(\cdot\), \textit{e.g.} a visual representation \(r\), a question \(q\) or batches thereof.}

\nomenclature[A, 02]{\(W\)}{A trainable weight tensor.}

\nomenclature[A, 03]{\(b\)}{A trainable additive bias tensor.}

\nomenclature[A, 02]{\(d_{\cdot}\)}{A scalar used to denote the feature dimension of a vector, matrix or tensor \(\cdot\).}


% Graphs
\nomenclature[G, 00]{\(\mathcal{G} = (\mathcal{V}, \mathcal{E})\)}{A directed graph.}

\nomenclature[G, 01]{\(\mathcal{V} = \{v_i \mid i \in \Z^+\}\)}{A set of vertices of a directed graph}

\nomenclature[G, 02]{\(\mathcal{E} = \{(v_i, v_j) \mid i \in \Z^+, j \in \Z^+\}\)}{A set of edges of a directed graph.}

\nomenclature[G, 03]{\(v\)}{A vertex of a graph \(\mathcal{G}\).}

% \nomenclature[G, 04]{\(\Vec{v}\)}{A node embedding vector corresponding to a vertex \(v \in \mathcal{V}\). {\color{red}TODO: Consider \(\Vec{h}_v\)}}

\nomenclature[G]{\(A \in \R^{\mid \mathcal{V}\mid \times \mid \mathcal{V} \mid}\)}{An adjacency matrix given a graph \(\mathcal{G}\) with vertices \(\mathcal{V}\).}

\nomenclature[G]{\(D \in \R^{\mid \mathcal{V}\mid \times \mid \mathcal{V} \mid}\)}{A degree matrix with diagonal elements \(D_{ii} = \sum_{j} A_{ij}\)}

\nomenclature[G]{\(\Tilde{A} = A + I_{\mid \mathcal{V} \mid}\)}{An adjacency matrix with added self-loops.}

