\chapter{Conclusion}

To conclude, I review the primary outcomes of each chapter in this dissertation, summarising how they relate to the visual question answering (VQA) research community, from the literature that inspired my model architecture through to final results and contributions made to the field, and potential directions for future research.

In \chapterautorefname{ \ref{chapter:literature}}, I dissected the VQA problem into three main areas of interest, namely dataset collection, model design and metric creation. I explored methods developed by various VQA dataset creators to collect questions and images in large quantities, whilst mitigating the problematic language biases that make the evaluation of VQA models difficult. I investigated the promising results of recent graph-based models for graph representation learning and VQA, motivated by the lack of current research regarding VQA on pre-annotated scene graphs. Finally, I summarised the types of metrics used for the evaluation of both multiple-choice and open-ended VQA tasks.

Armed with this knowledge, I introduced the methodology behind my VQA model in \chapterautorefname{ \ref{chapter:methodology}}, dividing it into four main components. Together, the question processing module, scene graph processing module, reasoning module and output module form an architecture for single-concept open-ended VQA tasks

\section{Future Work}

\begin{itemize}
  \item Scene graph generation
  \item Edge convolutions for relations
\end{itemize}
